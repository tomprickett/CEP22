%%
%% This is file `sample-sigconf.tex',
%% generated with the docstrip utility.
%%
%% The original source files were:
%%
%% samples.dtx  (with options: `sigconf')
%% 
%% IMPORTANT NOTICE:
%% 
%% For the copyright see the source file.
%% 
%% Any modified versions of this file must be renamed
%% with new filenames distinct from sample-sigconf.tex.
%% 
%% For distribution of the original source see the terms
%% for copying and modification in the file samples.dtx.
%% 
%% This generated file may be distributed as long as the
%% original source files, as listed above, are part of the
%% same distribution. (The sources need not necessarily be
%% in the same archive or directory.)
%%
%% The first command in your LaTeX source must be the \documentclass command.
\documentclass[sigconf]{acmart}
\usepackage[true]{anonymous-acm}
\settopmatter{authorsperrow=3}
%% NOTE that a single column version may be required for 
%% submission and peer review. This can be done by changing
%% the \doucmentclass[...]{acmart} in this template to 
%% \documentclass[manuscript,screen]{acmart}
%% 
%% To ensure 100% compatibility, please check the white list of
%% approved LaTeX packages to be used with the Master Article Template at
%% https://www.acm.org/publications/taps/whitelist-of-latex-packages 
%% before creating your document. The white list page provides 
%% information on how to submit additional LaTeX packages for 
%% review and adoption.
%% Fonts used in the template cannot be substituted; margin 
%% adjustments are not allowed.
%%
%%
%% \BibTeX command to typeset BibTeX logo in the docs
\AtBeginDocument{%
  \providecommand\BibTeX{{%
    \normalfont B\kern-0.5em{\scshape i\kern-0.25em b}\kern-0.8em\TeX}}}

%% Rights management information.  This information is sent to you
%% when you complete the rights form.  These commands have SAMPLE
%% values in them; it is your responsibility as an author to replace
%% the commands and values with those provided to you when you
%% complete the rights form.
\setcopyright{acmcopyright}
\copyrightyear{2022}
\acmYear{2022}
\acmDOI{10.1145/1122445.1122456}

%% These commands are for a PROCEEDINGS abstract or paper.
\acmConference[CEP '22]{CEP '22: ACM Computing Education Practice}{January 06, 2022}{Durham, UK}
\acmBooktitle{CEP '22: ACM Computing Education Practice Conference,
  January 06, 2022, Durham, UK}
\acmPrice{15.00}
\acmISBN{978-1-4503-XXXX-X/18/06}


%%
%% Submission ID.
%% Use this when submitting an article to a sponsored event. You'll
%% receive a unique submission ID from the organizers
%% of the event, and this ID should be used as the parameter to this command.
%%\acmSubmissionID{123-A56-BU3}

%%
%% The majority of ACM publications use numbered citations and
%% references.  The command \citestyle{authoryear} switches to the
%% "author year" style.
%%
%% If you are preparing content for an event
%% sponsored by ACM SIGGRAPH, you must use the "author year" style of
%% citations and references.
%% Uncommenting
%% the next command will enable that style.
%%\citestyle{acmauthoryear}

%%
%% end of the preamble, start of the body of the document source.
\begin{document}

%%
%% The "title" command has an optional parameter,
%% allowing the author to define a "short title" to be used in page headers.
%\title{National Developmental Support for Early Career Academics in the UK Computer Science Community}
\title{Co-Constructing a Community of Practice for Early-Career Computer Science Academics in the UK}

%%
%% The "author" command and its associated commands are used to define
%% the authors and their affiliations.
%% Of note is the shared affiliation of the first two authors, and the
%% "authornote" and "authornotemark" commands
\authoranon{
  \author{Tom Crick}
  \orcid{0000-0001-5196-9389}
\authornote{{\emph{N.B.}} all authors contributed equally to this paper}
\affiliation{
	\institution{Swansea University}
	\city{Swansea}
	\country{UK}
}
\email{thomas.crick@swansea.ac.uk}

\author{James H. Davenport}
\authornotemark[1]
\affiliation{%
	\institution{ University of Bath}
	\city{Bath}
	\country{UK}
}
\email{masjhd@bath.ac.uk}

\author{Paul Hanna}
\authornotemark[1]
\affiliation{%
	\institution{ Ulster University}
	\city{Belfast}
	\country{UK}
}
\email{jrp.hanna@ulster.ac.uk}

\author{Alan Hayes}
\authornotemark[1]
\affiliation{%
	\institution{ University of Bath}
	\city{Bath}
	\country{UK}
}
\email{ah347@bath.ac.uk}

\author{Alastair Irons}
\authornotemark[1]
\affiliation{
	\institution{ University of Sunderland}
	\city{Sunderland}
	\country{UK} }
\email{alastair.irons@sunderland.ac.uk}

\author{Keith Miller}
\authornotemark[1]
\affiliation{
	\institution{ Manchester Metropolitan University}
	\city{Manchester}
	\country{UK} }
\email{k.miller@mmu.ac.uk}

\author{Tom Prickett}
\authornotemark[1]
\affiliation{
	\institution{ Northumbria University}
	\city{Newcastle upon Tyne}
	\country{UK}
}
\email{tom.prickett@northumbria.ac.uk}

\author{Rupert Ward}
\authornotemark[1]
\affiliation{
	\institution{University of Huddersfield}
	\city{Manchester}
	\country{UK} }
\email{rupert.ward@hud.ac.uk}

\author{Becky Allen}
\authornotemark[1]
\affiliation{
	\institution{University of Sunderland}
	\city{Sunderland}
	\country{UK} }
      \email{becky.allen@sunderland.ac.uk}

      \author{Bhagyashree Patil}
\authornotemark[1]
\affiliation{
	\institution{University of Bath}
	\city{Bath}
	\country{UK} }
      \email{bp397@bath.ac.uk}

            \author{Simon Payne}
\authornotemark[1]
\affiliation{
	\institution{University of South Wales}
	\city{Pontypridd}
	\country{UK} }
\email{simon.payne@southwales.ac.uk}
}

%%
%% By default, the full list of authors will be used in the page
%% headers. Often, this list is too long, and will overlap
%% other information printed in the page headers. This command allows
%% the author to define a more concise list
%% of authors' names for this purpose.
\renewcommand{\shortauthors}{\textanon {Crick et al.} {BLINDED}}

%%
%% The abstract is a short summary of the work to be presented in the
%% article.
\begin{abstract}
Early-career academics across all disciplines in the UK face
significant challenges, and computer science is no exception. There
are challenges in terms of developing an independent research career,
delivering high quality learning and teaching, maintaining their own
professional development, as well as wider academic service
commitments. Tertiary education institutions in the UK often provide
support through some combination of mentoring, coaching, and
training. Early-career faculty often have to work towards professional
recognition of their teaching, either by direct application or via
successful completion of an accredited institutional taught
postgraduate course. This paper reports on progress towards
supplementing institutional-level support through an evolving UK-wide
initiative, co-constructed with early-career academics, to build
diverse and resilient communities of practice in computer
science. Insights are provided as to how the initiative supplements
current institutional approach and is underpinned by national-level
academic practice developmental events, professional body engagement,
alongside cross-institutional mentoring and buddying schemes.
\end{abstract}

%%
%% The code below is generated by the tool at http://dl.acm.org/ccs.cfm.
%% Please copy and paste the code instead of the example below.
%%
\begin{CCSXML}
<ccs2012>
   <concept>
       <concept_id>10003456.10003457.10003527</concept_id>
       <concept_desc>Social and professional topics~Computing education</concept_desc>
       <concept_significance>500</concept_significance>
       </concept>
 </ccs2012>
\end{CCSXML}

\ccsdesc[500]{Social and professional topics~Computing education}


%%
%% Keywords. The author(s) should pick words that accurately describe
%% the work being presented. Separate the keywords with commas.
\keywords{Early-career academics, community of practice, professional
  development, co-construction}


%%
%% This command processes the author and affiliation and title
%% information and builds the first part of the formatted document.
\maketitle

\section{What is it?}	
%%A short description of the practice you're presenting
\label{sec:What}
This paper reports on the progress to date of an emerging initiative
to support the development needs of early career academics in the
United Kingdom. The background analysis undertaken to initiate this
scheme and the outcomes of a related workshop at \textanon{ACM UK \&
Ireland Computing Education Research Conference 2020
(UKICER’20)}{BLINDED CONFERENCE NAME} to formulate a pilot course are
discussed \textanon{ a previous paper
\cite{CricketAl2021CEP}}{[BLINDED]}. The initiative has three key
activities: ({\emph{i}}) Developmental/training sessions;
({\emph{ii}}) Cross-university mentoring; and ({\emph{iii}})
Cross-university buddying. Initially, the steering group set out with
the goal of having two key activities, namely ({\emph{i}}) and
({\emph{ii}}) above, but as part of the co-construction process with participants, buddying
was added as a third goal. Universities from across the four nations
of the UK have been involved and to date 59 early career colleagues
from 16 different institutions have participated in the scheme.

The scheme was formally initiated in December 2020. To date three
developmental/networking events have been delivered online, primarily
due to the impacts of the ongoing COVID-19 pandemic, which has
presented considerable challenges for computer
science~\cite{crick-et-al:ukicer2020,wg1:iticse2021} and for higher
education in
general~\cite{watermeyer-et-al:he2020,CrickCovidUK}. Alternative
approaches would have been considered in other circumstances. However
many of the participants at the workshops highlighted that due to
competing work-pressures and expense issues they preferred the adopted
virtual format. Following each event, feedback was sought by a
post-event survey; the outcomes of these surveys are explored in
section~\ref{Sec:DoesItWork}.

The first event took place in December 2020 attracting 22 attendees
from seven different universities representing all four nations of the
UK. Four main activities were provided: ({\emph{a}}) Challenges and
tools for teaching programming exploring tools for automated testing
and plagiarism detection and provided good practice examples for
discussion; ({\emph{b}}) Supervising CS project students which was an
interactive session related to the challenges and opportunities of
supervising CS project students; ({\emph{c}}) Prior to the event,
attendees were asked to pose three questions for a panel of five
experienced CS professors to address; and ({\emph{d}}) A workshop that
explored how could the scheme help/support the participants, how could
a diverse, resilient and sustainable community be developed for the
participants, and did the format work and what could be improved?

The second event took place in March 2021; again there were 22
attendees from across all four nations of the UK. This event was
designed to be more
interactive in approach. The main activities were: ({\emph{a}})
Networking Opportunities with breakout rooms being used for the
attendees to discuss the challenges and successes they have been
experiencing and how this initiative could best support them;
({\emph{b}}) Professional Bodies and Accreditation related to Computer
Science were explored; ({\emph{c}}) 'Would you like us to set up a
mentoring scheme?' was explored; and ({\emph{d}}) the existing
information sharing opportunities were discussed i.e. the related
conferences, journal club, available training, etc.

The third event took place in May 2021 attracting 15 colleagues. Given
the timing in the UK academic year, the focus of this event was
delivering effective higher education assessment and feedback
processes. This session was led by former employee of Advanced
HE. Advance HE is a member-led, sector-owned charity that works with
the higher education sector across the world to enhance higher
education for staff, students and society. Among other activities,
Advanced HE provides the de facto standard for accrediting educational
competence for UK Higher Education i.e. FHEA.

As part of these events, there has been an ongoing discussion
regarding a mentoring scheme and how it would operate. The
expectations for mentoring have been agreed as: the mentoring is
external i.e. the mentor and mentee do not work for the same
university; the time commitment is initially 60 minutes, four times
per year; an agreed focus is taken (education, research, career,
sub-discipline area, professional registration e.g. FHEA/SFHEA, NTFS,
MBCS/CITP/FBCS, or other agreed focus); there is a process for
matching mentors and mentees, with an initial meeting to confirm
suitability; and there is an expectation that the date of mentoring
meetings is recorded. The first batch of mentor/mentees was assigned
in October 2021.

Alongside the discussion regarding mentoring, at the second and third
event there was a discussion regarding buddying. The preference from
the participants, was that buddying should not be one-on-one, but that
small groups of buddies be established. As with mentoring, there is an
expectation to record the date of meetings. The intention is that as
more participants join the buddying scheme, thematic groups can be
formed.

Over the course of the initiative, the steering group has also
expanded. From an initial 12 academics representative of all the home
nations of the UK and a variety of different university types, the
steering group now consists of 24 academics representing 20
universities (England: 16, Wales: 4, Scotland: 3 and NI: 1).

\section{Why are you doing it?}
%%What happened before? What is it changing / replacing / improving? What gap is it filling?
\label{sec:Why}
Starting out in your academic career can be
challenging~\cite{Thomas2015} and potentially lonely
~\cite{Foote2009}. Many new academics have moved on from either funded
PhD studentships or postdoctoral research positions in which they have
the luxury of placing a primacy in their research. Others join
universities from industrial careers and hence find themselves in the
challenging position of establishing a research portfolio alongside
their learning and teaching activities. All face the challenge of
balancing delivering high quality education, growing their research
profile, and completing wider professional service commitments. For
many this is while working in a precarious and for some a short term
contract \cite{UCU,JaffeS}. As a backdrop to this, workload in the UK
higher education sector has become a highly contested
issue~\cite{UCU2016} and a common topic in many discussions with
early-career practitioners (and more so with the impact of
COVID-19~\cite{crick-et-al:ukicer2020,watermeyer-et-al:he2020,CrickCovidUK}).

Making this transition requires learning. The quality of learning
support provided will be promoted in part by the strength of the
community of practice operating within the
department~\cite{Bolander2008} and the communities of practice that
exist at a national and international
level~\cite{Thomas2015}. Furthermore, this can and should be
co-constructed with early-career academics; we refer to
co-construction as the joint creation of an action, activity,
identity, institution, or other culturally meaningful
reality~\cite{jacoby+ochs:1995}. The ``{\emph{co-}}'' prefix is
intended to cover a range of interaction processes, including
collaboeration, cooperarion and coordination. Indeed, this body of
work, and this paper, has been co-constructed with early-career
colleagues.

In computing education, there are a number of discipline-specific
challenges that have been discussed in the literature, especially at
university-level. For example, the teaching of introductory
programming effectively has persistent issues
~\cite{davenport-et-al:latice2016,murphy-et-al:programming2017,simon-et-al:sigcse2018},
attrition and failure rates can be high, with a range of issues
impacting barriers to
progression~\cite{Watson:2014:FRI:2591708.2591749}. Student
satisfaction as measured by satisfaction surveys is reported as
commonly below that of other disciplines \cite{Sinclair2015} and
varies across the discipline with some subdiscipline areas facing
particular challenges to navigate \cite{Knutas2021}. Discipline
related challenges linked to delivering teamwork are also reported
\cite{Gordon2010,Phillips2021}. The employment prospects of graduates
from some computing related degrees have been reported as inferior to
other disciplines\cite{shadbolt2016shadbolt}. The appropriate handling
of gender inclusion \cite{Winter2021} and neurodiversity
\cite{Stuurman2109} remain discipline challenges. Addressing these
challenges effectively requires the development of specialised
educational competencies, which commonly need to be developed,
alongside enhancing skills, reputation and outputs within an
academic’s discipline specialism.

Together these pressures highlight that early career academics could
potentially benefit from further support from the wider
community. Offering developmental events is a tangible way of
providing assistance where it is needed. Mentoring has also become a
commonly recognised approach that can contribute to the professional
development of academics. Indeed such schemes are very common in
universities and departments. Typically an early-career colleague
would be mentored by an experienced academic who normally is not their
direct line manager. It has been reported that such schemes can help
diversify the staff base \cite{Golubchik2018}. However, restricting
guidance and support to within one university rather than a wider
community has its limitations \cite{Golubchik2018}, so a
department-based mentoring scheme does not replace wider community
support.  It is argued that community-based mentoring offers
additional benefits through being impartial and by allowing space for
open discussion not linked to line management. It is recommended that
early-career colleagues use community-based mentoring to gain access
to a wider discipline-based pool of knowledge.

Use of buddying schemes for learners in higher education is commonly
reported to be beneficial, for example~\cite{Hayes2020,May20}. Use of
buddying between academic colleagues is less well reported. Buddying
has been reported as a supporting mechanism to help support the
onboarding of expatriate academics to a particular
university~\cite{Wilkins2019}. Attempting to establish a nationwide,
cross university scheme presents a new departure. The genesis for such
a scheme came from the early-career colleagues themselves and it has
been configured entirely around their suggestions.
	
\section{Where does it fit?}
%%A short description of your teaching context. You may, for instance, include a description of intake, class size, curriculum sequence; anything that's necessary for others to understand your situation. How do things work at your institution?

There are a number of group who can benefit from this work.
scheme. To date, there have principally been three key participant
groups: {\emph{Early-career lecturers}} who have recently been ap-
pointed to an academic post. These may have teaching and research or
alternatively more teaching focused responsibilities. Some have joined
from industry, others from a research background; {\emph{Aspiring
academics}} who are typically PhD students or post-docs and are
aspiring towards a full academic role; and {\emph{more
established/senior colleagues}} who are new to UK higher education
and hence are seeking help to acclimatise. There is some
variety too in the colleagues who are supporting the initiative. All
have had a degree of seniority either via their presence in the
computing education community or the responsibilities they adopt
within their own university and/or nationally. There has been
significant representation from members of the professoriate but not
exclusively so.

\section{Does it work?}	
%%How do you know? Give some evidence of effectiveness in context
\label{Sec:DoesItWork}
The scheme has been run as a trial/prototype for a small number of UK
universities. This was deliberate in order to establish the
feasibility of the approach, and facilitate the co-construction, to
allow the scheme to evolve and develop in response to the voice of the
participants. Of note is that there are a number of participants who
have actively engaged with all the events to date. There is also a
growing number of requests, commonly from peers at the university of
attendees to join the scheme.

Anonymous post-event surveys have also been used for evaluation. The
first event was well received with 11 participants completing the post
event survey. When asked ``{\emph{Overall, was the workshop
useful}}'', four attendees strongly agreed and seven agreed. One item
of constructive feedback received was the session could be even more
interactive which was taken on board for the second event.

The second event was again well received; of particular note was the
strength of positive feeling related to the networking
opportunities. Also another outcome was the suggestion that buddying
should be considered as a possibility alongside mentoring. In terms of
post-event feedback, there were eight respondents, six of whom
strongly agreed that ``{\emph{Overall, the workshop was useful}}'' and
a further two agreed.

For the third event, only two responses to the survey were
received; again, these were positive. Other feedback indicated this was a
very busy time of year. It is also noted the session was less computer
science specific than the previous events. The session was scheduled
when many colleagues would be engaged in marking, which was deliberate
so the activity could be supported. On reflection attendance may have
been higher at another time of year. These factors may all have made
an impact and will be considered for future events. As with the second
event, considerable use of break out rooms was made, to enable more
interaction and facilitate opportunities for networking.

\begin{comment}
TO DO - evaluation of Mentoring - PH what is needed here? Is sufficient to indicate the first pilot of 10 Mentees has been established? Or do we need feedback from the participants?
\end{comment}

Over summer 2021, volunteer mentors and expressions of interest for
mentoring were sought. This resulted in 12 expressions of interest in
having a mentor. Many colleagues seek mentoring in more than one area:
92\% are seeking support with research, 75\% with career and career
planning, 58\% in education, 57\% in their subdiscipline area and 50\%
in professional membership and registration. Initial mentoring
relationships were established in September 2021 with first meetings
scheduled to take place in October 2021. This is being closely
monitored to view the emerging practice and to better understand how
this can be supported, promoted and scaled.

\begin{comment}
To DO - evaluation of Buddying - PH what is needed here? Is sufficient to indicate the first pilot of 10 buddies has been established? Or do we need feedback from the participants?
\end{comment}

The buddying scheme pilot began in July 2021; an initial group of five
buddies has been meeting regularly since then to discuss items of
common interest. In October 2021, a second group of five buddies was
formed and is progressing similarly. Both groups are being closely
monitored to view the emerging practice and to better understand how
these groups can be supported.


\section{Who else has done this?}
%%Where did you get the idea from? (If from published reports, please include references). How did you find out about it? Was it easy/hard to adopt? What did you change?
As discussed in section~\ref{sec:Why}, in addition to wider
educational challenges (including the ongoing impact of COVID-19),
many disciplines including computing have a range of
discipline-specific challenges. The limitations of generic
institutional schemes to address the educational challenges of physics
has been reported~\cite{Magueijo2009}. Mathematics is one such
discipline and one professional body (the Institute of Mathematics and its
Applications) has previously run courses for early-career colleagues to help
establish them in the discipline~\cite{IMA}.

Peer-to-peer conversations have been reported to be a commonly used
mechanism for professional development~\cite{King2004} and a number of
national and international communities exist to help promote such
conversations. Internationally, groups such as ACM SIGCSE or the IEEE Education
Society promote this dialogue via conferences and other activities.
In the UK and Ireland a SIGCSE chapter further promotes these
discussions by running two annual conferences, one focusing on
practice (CEP) and the other on educational research (UKICER).

% Much of these conversations
% support the professional development of colleagues who specialise in
% computing education. Whilst some early career colleagues will actively
% engage with this speciality, all will be required to develop
% educational capacity and effectiveness.
 
Training programmes are run by individual higher education
providers. Additionally, in the UK, Advance HE deliver training
programmes for academics at different stages of their
career~\cite{HEATraining}. However this training is not discipline
specific. The Council of Professors and Heads of Computing (CPHC) run
occasional workshops in a variety of issues, for example the ``Chair
in 10 Years'' workshop which is aimed at facilitating career planning
and ``New Head of Department'' workshop. Whilst these are well
received contributions, it is clear that developmental needs are
broader than those supported by these workshops.

%%TO DO - Add LTSN from Alan here

\section{What will you do next?}
%%Will you vary this, or develop it further?
\begin{comment}

Scaling up


Funding 


Shared CPD learning materials


Address EDI issues in computer science


Reverse mentoring


Align with BCS


Special interest group

Web space for repository and functionality (mentoring, buddying) to store CPD sessions, shared resources for new CS academics, examples of good practice and a discussion space.

Explore what BCS are doing with professionals in non academic roles

Encourage research on pedagogy and educational issues in CS

Examine if programme framework is transferrable to other disciplines
\end{comment}

Firstly, to be sustainable and to scale up the programme {\emph{a)}}
so it is available to a much wider population and {\emph{b)}} has
repeating components there is a need to move to a product which has
the capacity to manage larger cohorts and a greater volume of
material. This will require sustainable funding. Secondly, the
aspiration is to provide a searchable repository (developmental
sessions, shared resources, and examples of good practice) from the
workshops. Thirdly, the team has worked with the UK's professional
body for the computing and the IT industry (BCS, The Chartered
Institute for IT), who can host the repository and facilitate
mentoring and buddying schemes the project is trying to establish. The
intention is to establish a new Special Interest Group (SIG) within
the BCS~\cite{BCSSIG} and at the time of writing initial approval has
been granted. Fourthly, the initiative wishes to help address
equality, diversity and inclusivity (EDI) issues in computer
science. EDI issues are key challenges for everyone working in higher
education and computer science is no exception. There are specific
related challenges in computing -- both in computing education and for
the subject itself (such as accessibility of systems and digital
poverty). Supporting early career colleagues to identify and address
EDI challenges and embedding into good practice is a critical aspect
of the project. Fifthly, it is recognised that there is a virtuous
circle between computing education research and sustained improvement
in computing education. It has also been argued that ``the rapidly
evolving nature of computing together with changing educational
technologies encourages continuous review of the pedagogy for
computing courses''~\cite{Iroms2004}. Progress in establishing
computer science educational research has been slow and whilst there
have been notable examples of excellence in this space, universal
adoption and universal acceptance has not happened. This programme
will promote educational research to early career colleagues and
thereby help to establish it as a mainstream thread in computer
science research. Finally, the project wishes to examine if the
programme framework is transferable to other disciplines. One of the
aspirations is to use the work we are doing in computer science to
create a generic framework which will allow other STEM, near-STEM and
potentially non-STEM subjects to contextualise subject specific
programmes for their new academics.


\section{Why are you telling us this?}
In the UK, the communities of practice related to computing education
have rapidly evolved in recent years. The initiative discussed in this
paper further develops this work by: ({\emph{i}}) creating networking
and developmental events aimed specifically at introducing early
career colleagues to one another and the wider community of computing
education practice; ({\emph{ii}}) establishing co-constructed national
mentoring programmes, running across universities providing support
for early career colleagues development in research and education; and
({\emph{iii}}) providing scaffolding for nascent professional networks
across universities by the establishment of a nationwide buddying
scheme. The initiative presents significant opportunities for those at
differing stages of their career, co-constructed with a diverse groups
of academics.  For those at an early career stage, the initiative
presents a further source of beneficial professional development. For
those with more experience, it gives an opportunity to better
understand challenges early career colleagues face as well as to grow
their professional networks and foster the emerging community of
computing education practice in the UK.

% \begin{acks}
	
% 	The authors would like to thank the members of the wider computing education community who have supported the initiative either be contributing to events, attending events, providing feedback, mentoring or being a mentee or a buddy.
% 	The BCS Academy of Computing has supported the development of this paper
	
	
% \end{acks}


%%
%% The next two lines define the bibliography style to be used, and
%% the bibliography file.
\bibliographystyle{ACM-Reference-Format}
\bibliography{NewLecturer}



\end{document}
\endinput
%%
%% End of file `sample-sigconf.tex'.
