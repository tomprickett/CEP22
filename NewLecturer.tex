%%
%% This is file `sample-sigconf.tex',
%% generated with the docstrip utility.
%%
%% The original source files were:
%%
%% samples.dtx  (with options: `sigconf')
%% 
%% IMPORTANT NOTICE:
%% 
%% For the copyright see the source file.
%% 
%% Any modified versions of this file must be renamed
%% with new filenames distinct from sample-sigconf.tex.
%% 
%% For distribution of the original source see the terms
%% for copying and modification in the file samples.dtx.
%% 
%% This generated file may be distributed as long as the
%% original source files, as listed above, are part of the
%% same distribution. (The sources need not necessarily be
%% in the same archive or directory.)
%%
%% The first command in your LaTeX source must be the \documentclass command.
\documentclass[sigconf]{acmart}
\usepackage[true]{anonymous-acm}
\settopmatter{authorsperrow=3}
%% NOTE that a single column version may be required for 
%% submission and peer review. This can be done by changing
%% the \doucmentclass[...]{acmart} in this template to 
%% \documentclass[manuscript,screen]{acmart}
%% 
%% To ensure 100% compatibility, please check the white list of
%% approved LaTeX packages to be used with the Master Article Template at
%% https://www.acm.org/publications/taps/whitelist-of-latex-packages 
%% before creating your document. The white list page provides 
%% information on how to submit additional LaTeX packages for 
%% review and adoption.
%% Fonts used in the template cannot be substituted; margin 
%% adjustments are not allowed.
%%
%%
%% \BibTeX command to typeset BibTeX logo in the docs
\AtBeginDocument{%
  \providecommand\BibTeX{{%
    \normalfont B\kern-0.5em{\scshape i\kern-0.25em b}\kern-0.8em\TeX}}}

%% Rights management information.  This information is sent to you
%% when you complete the rights form.  These commands have SAMPLE
%% values in them; it is your responsibility as an author to replace
%% the commands and values with those provided to you when you
%% complete the rights form.
\setcopyright{acmcopyright}
\copyrightyear{2022}
\acmYear{2022}
\acmDOI{10.1145/1122445.1122456}

%% These commands are for a PROCEEDINGS abstract or paper.
\acmConference[CEP '22]{CEP '22: ACM Computing Education Practice}{January 06, 2022}{Durham, UK}
\acmBooktitle{CEP '22: ACM Computing Education Practice Conference,
  January 06, 2022, Durham, UK}
\acmPrice{15.00}
\acmISBN{978-1-4503-XXXX-X/18/06}


%%
%% Submission ID.
%% Use this when submitting an article to a sponsored event. You'll
%% receive a unique submission ID from the organizers
%% of the event, and this ID should be used as the parameter to this command.
%%\acmSubmissionID{123-A56-BU3}

%%
%% The majority of ACM publications use numbered citations and
%% references.  The command \citestyle{authoryear} switches to the
%% "author year" style.
%%
%% If you are preparing content for an event
%% sponsored by ACM SIGGRAPH, you must use the "author year" style of
%% citations and references.
%% Uncommenting
%% the next command will enable that style.
%%\citestyle{acmauthoryear}

%%
%% end of the preamble, start of the body of the document source.
\begin{document}

%%
%% The "title" command has an optional parameter,
%% allowing the author to define a "short title" to be used in page headers.
\title{National Developmental Support for Early Career Academics in the UK Computer Science Community}

%%
%% The "author" command and its associated commands are used to define
%% the authors and their affiliations.
%% Of note is the shared affiliation of the first two authors, and the
%% "authornote" and "authornotemark" commands
\authoranon{
\author{Tom Crick}
\authornote{{\emph{N.B.}} all authors contributed equally to this paper}
\affiliation{
	\institution{Swansea University}
	\city{Swansea}
	\country{UK}
}
\email{thomas.crick@swansea.ac.uk}

\author{James H. Davenport}
\authornotemark[1]
\affiliation{%
	\institution{ University of Bath}
	\city{Bath}
	\country{UK}
}
\email{j.h.davenport@bath.ac.uk}

\author{Alan Hayes}
\authornotemark[1]
\affiliation{%
	\institution{ University of Bath}
	\city{Bath}
	\country{UK}
}
\email{ah347@bath.ac.uk}

\author{Paul Hanna}
\authornotemark[1]
\affiliation{%
	\institution{ Ulster University}
	\city{Belfast}
	\country{UK}
}
\email{jrp.hanna@ulster.ac.uk}


\author{Alastair Irons}
\authornotemark[1]
\affiliation{
	\institution{ Sunderland University}
	\city{Sunderland}
	\country{UK} }
\email{alastair.irons@sunderland.ac.uk}

\author{Keith Miller}
\authornotemark[1]
\affiliation{
	\institution{ Manchester Metropolitan University}
	\city{Manchester}
	\country{UK} }
\email{k.miller@mmu.ac.uk}

\author{Tom Prickett}
\authornotemark[1]
\affiliation{
	\institution{ Northumbria University}
	\city{Newcastle upon Tyne}
	\country{UK}
}
\email{tom.prickett@northumbria.ac.uk}

\author{Rupert Ward}
\authornotemark[1]
\affiliation{
	\institution{University of Huddersfield}
	\city{Manchester}
	\country{UK} }
\email{rupert.ward@hud.ac.uk}

}

%%
%% By default, the full list of authors will be used in the page
%% headers. Often, this list is too long, and will overlap
%% other information printed in the page headers. This command allows
%% the author to define a more concise list
%% of authors' names for this purpose.
\renewcommand{\shortauthors}{\textanon {Crick et al.} {BLINDED}}

%%
%% The abstract is a short summary of the work to be presented in the
%% article.
\begin{abstract}
Significant challenges are presented to early career academics in the United Kingdom and other jurisdictions in all disciplines and computer science academic are no exception.There are challenges in
terms of excelling at research, delivering high quality education, wider academic service commitments, as well as their
own professional development. Higher Education Institutions provide support for this by a combination of mentoring, coaching and training with early career faculty often  working towards Fellowship of the Higher Education Academy (now known as Advance HE), either by direct
application or via successful completion of an accredited
institutional taught postgraduate course). This paper reports upon progress to supplement this support from within Computing departments with an evolving UK-wider initiative to provide developmental events and to establish cross higher education institute buddying and mentoring schemes.
This paper presents a progress report regarding how this initiative has proceeded and explores the future directions under consideration, this is presented provides insights to how this initiative supplements current formal
institutional requirements with  national-level academic
practice developmental opportunities alongside national professional mentoring and buddying.

TO DO : Should we be using "Higher Education Institute" or "University"?

\end{abstract}

%%
%% The code below is generated by the tool at http://dl.acm.org/ccs.cfm.
%% Please copy and paste the code instead of the example below.
%%
\begin{CCSXML}
<ccs2012>
   <concept>
       <concept_id>10003456.10003457.10003527</concept_id>
       <concept_desc>Social and professional topics~Computing education</concept_desc>
       <concept_significance>500</concept_significance>
       </concept>
 </ccs2012>
\end{CCSXML}

\ccsdesc[500]{Social and professional topics~Computing education}


%%
%% Keywords. The author(s) should pick words that accurately describe
%% the work being presented. Separate the keywords with commas.
\keywords{Early-career academics, community of practice, professional development, computer science education}


%%
%% This command processes the author and affiliation and title
%% information and builds the first part of the formatted document.
\maketitle

\section{What is it?}	
%%A short description of the practice you're presenting
\label{sec:What}
This paper reports upon progress made during the first year of operation of a national initiative supporting the development needs of early career academics in the United Kingdom. The original and background analysis taken to initiate this scheme including the recruitment of a steer-group of academics from across a representative range of UK Universities and the outcomes of a related workshop at \textanon{ACM UK \& Ireland Computing Education Research Conference 2020 (UKICER’20)}{BLINDED CONFERENCE NAME} to formulate a pilot course are discussed in \textanon{ a previous paper \cite{CricketAl2021CEP}}{BLINDED}.The resulting initiative has three key activities: 

\begin{enumerate}
	\item Developmental sessions
	\item Cross Higher Educational Institute mentoring
	\item Cross Higher Educational Institute buddying
\end{enumerate}

The steering group set out with the goal of having two key activities (1) and (2) but following input from participants a third goal (3) was added. Higher Educational Institutes from across the four nations of the UK have been involved and to date 52 early career colleagues from NNN different Institutes have participated in the scheme. This paper provides an update upon the progress made to date.

The scheme was formally started in December 2020. To date three developmental / networking events have been ran. These events all took place online via Microsoft Teams. The original decision to operate in this manner was partly because of the ongoing Covid-19 pandemic, which has presented considerable challenges for the operation of higher education in general \cite{CrickCovidUK,CrickCovidAus2021,CrickCovidPoster} and alternative means may have been considered in other circumstances. However many of the participants at the workshops highlight that due competing work-pressures and expense issues the adopted virtual format was preferred by many. Following each event, feedback was sought by a post event survey. The outcomes of these surveys is briefly explored in \ref{Sec:DoesItWork}.

The first event took place in December 2020. There were 22 attendees from 7 different universities representing all four nations of the UK. Four main activities were provided:
\begin{enumerate}
	\item Challenges and tools for teaching programming. It explored tools for automated testing and plagiarism detection and provided good practice examples for discussion.
	\item Supervising CS Project Students.This was an interactive session related to the challenges and opportunities of supervising CS project students
	\item Prior to the the event attendees were asked to pose 3 questions for a panel of 5 Professors to address. 
	\item A workshop explored
		\begin{enumerate}
			\item How can the scheme support / help the participants
			\item How can a community be developed for the participants
			\item Did the format work and what would work better
		\end{enumerate}
\end{enumerate}

The second event took place in March 2021.There were 22 attendees representing all four nations of the UK. This event was more interactive in approach. The main activities were:
\begin{enumerate}
	\item Networking Opportunities. Breakout rooms were used for the attendees to discuss the challenges and successes they have been experiencing and how this initiative could best support them
	\item Professional Bodies and Accreditation related to Computer Science were explored
	\item 'Would you like us to set up a mentoring scheme?' was explored
	\item The existing information sharing opportunities were discussed i.e. the related conferences, journal club, available training, etc. 
\end{enumerate}
The second event gave rise to the suggestion that a national buddying scheme may provided alternative benefits to a mentoring scheme.

The third event took place in May 2001. Given where it was in the academic calender in the United Kingdom, the focus of this event was Delivering Effective Higher Education Assessment and Feedback Processes. This session was lead by former employee of Advanced HE. Advance HE is a member-led, sector-owned charity that works with institutions and higher education across the world to improve higher education for staff, students and society. Among other activities Advanced HE provides the de facto standard for accrediting educational competence for UK Higher Education i.e. Fellowship of Higher Education (FHEA).  It was attended by 15 colleagues


As part of the events, there has been a running discussion regarding a mentoring scheme and how it would operate. The expectations for mentoring have been agreed to be: the mentoring is external i.e. the mentor and mentee do not work for the same university; the time commitment is initially 60 minutes, 4 times a year; an agreed focus is taken (education, research, career, sub-discipline area, professional registration (NTFS / FHEA / SFHEA / BCS CITP MBCS FBCS, etc), or other agreed focus); there is a process for matching mentors and mentees, with an initial meeting to confirm the suitability; and there is an expectation that the date of mentoring meetings. The first batch of mentor / mentees was assigned in September 2021. 

Alongside the discussion regarding mentoring, at the 2nd and 3rd event there was a discussion regarding buddying. The preference from the participants, was that buddying should not be one-on-one, instead small groups of buddies will be established. As with mentoring, there is an expectation to record the data of meetings. The intention is as more participants join the buddying scheme, them groups can be formed in more thematic areas. 

Over the course of the initiative, the steering group has also expanded. From an initial 12 academics representative of all the home nations of the UK and a variety of different university types, the steering group now consists of N academics from M different universities. Investigations have begun into the feasibility of establishing a BCS Specialist Interest Group \cite{BCSSIG} which would help support further future growth.




\section{Why are you doing it?}
%%What happened before? What is it changing / replacing / improving? What gap is it filling?
Becoming an established academic in UK higher education and elsewhere is challenging ~\cite{Thomas2015} and potentially lonely ~\cite{Foote2009}. Many new academics have move on from either PhD Studentships or Post-doc positions in which they have the luxury of placing a primacy in their research. Others join higher education institutes from industrial careers and hence find themselves in the challenging position of establishing a research portfolio. All face the challenge of balancing delivering high quality education, growing their research profile, and completing wider professional service commitments. For many this is while working in a precarious and for some upon a short term contract \cite{UCU,JaffeS}. As a back drop to this, workload in Higher Education has become a highly contested issue \cite{UCU2016}and a common topic in many discussions with early career colleagues.

Making this tranistion requires learning. The quality of learning support provided will be promoted in part by the strength of the community of practice operating within the department~\cite{Bolander2008} and the communities of practice that exist at a national and international level~\cite{Thomas2015}.

\label{sec:Why}

In the domain of computing education, there are a number of educational related discipline challenges that present further difficulties.  Effectively teaching introductory programming for the first time has persistent issues ~\cite{davenport-et-al:latice2016,murphy-et-al:programming2017,simon-et-al:sigcse2018}, attrition and failure rates can be high,with a range of issues impacting failure rates~\cite{Watson:2014:FRI:2591708.2591749}.  Student satisfaction as measured by satisfaction surveys is reported as commonly below that of other disciplines \cite{Sinclair2015}. Student satisfaction as measured by satisfaction surveys varies across the discipline with some subdiscipline areas facing particular challenges to navigate \cite{Knutas2021}. Discipline related challenges related to delivering team work are also reported \cite{Crick2020CEP,Gordon2010}. The employment prospects of graduates from some computing related degrees have been reported as inferior to other disciplines\cite{shadbolt2016shadbolt}. The appropriate handling of gender inclusion remains a discipline challenge \cite{Winter2021}. Effectively addressing these challenges requires the development of specialised educational competencies, which commonly need to be developed alongside enhancing skills, reputation and outputs within an academics discipline specialism. 

Together these pressures highlight part of the computing education community could potential benefit for further support from the wider community. Providing supporting developmental events is one way this support can be provided. Mentoring has also become a commonly recognised approach that can contribute to the development of academics. Indeed such schemes are very common in universities and departments. Typically an early-career colleague would be mentored by an experienced academic who normally is not their direct line manager. It has been reported that such schemes can help diversify the staff base \cite{Golubchik2018}. However, limiting guidance and support to one place rather than a community has its limitations \cite{deJanasz} so a mentoring scheme does not replace wider community support.

Use of buddying schemes for learners in higher education is commonly reported to be beneficial, for example~\cite{Hayes2020,May20}. Use of buddying between academic colleagues is less well reported. Buddying has been reported as a supporting mechanism to help support the onboarding of expatriate academics to a particular institute \cite{Wilkins2019}. Attempting to establish a nationwide, cross higher education institute scheme presents a new departure. The genesis for such a scheme came from the early-career colleagues themselves and it has been configured entirely around their suggestions.
	
\section{Where does it fit?}
%%A short description of your teaching context. You may, for instance, include a description of intake, class size, curriculum sequence; anything that's necessary for others to understand your situation. How do things work at your institution?
There is a large variation in terms of who may benefit from this scheme. To date, there are principally been three key participant groups
\begin{itemize}
	\item Early career lectures. These are colleagues who have recently been appointed to an academic post. These may have teaching and research or alternatively more teaching focused responsibilities. Some have joined from industry. Others from a research background.
	\item Aspiring academics. These are typically PhD students or post-docs who are aspiring towards a full academic role
	\item More senior colleagues who are new to the UK and hence are seeking help to acclimatise.
\end{itemize}

There is some variety too in the colleagues who are supporting the initiative. All have had a degree of seniority either via their presence in the computing education community or the responsibilities they adopt within their own institution and/or nationally. There has been significant representation from members of the proffesioriate but not exclusively so.





\section{Does it work?}	
%%How do you know? Give some evidence of effectiveness in context
\label{Sec:DoesItWork}
The scheme has been ran in a small prototype / trial manner for a relatively small number of high education institutes. This was deliberate in order to establish the feasibility of the scheme and allow the scheme to evolve in response to the voice of the participants. Of note is there are a number of participants who actively engage with all the events. There is also a growing number of requests, commonly from peers at the university of attendees to join the scheme.

In terms of the events, post-event surveys have been issued. The first event was well received with 11 participants completing the post event survey. When asked "Overall, was the workshop useful", 4 attendees strongly agreed and 7 agreed. One item of constructive feedback received was the session could be even more interactive which was taken on board for the second event. Also it was noted that some colleagues would have found attending a face to face event challenging which suggests that the online approach forced by the global pandemic may well be a good choice!

The second event was again well received. Of particular note was the strength of positive feeling related to the networking opportunities. Also another outcome was the suggestion that buddying should be considered as a possibility alongside mentoring. In terms of post-event feedback, there were 8 respondents 6 of whom strongly agreed that "Overal, the workshop was useful" and a further 2 agreed.

For the third event, sadly only 2 responses to the feedback. These were positive. Other feedback indicated this was a very busy time of year. It is also noted the session was less computer science specific than the previous events. The session ran during a period of time when many colleagues would be engaged in marking. This was deliberate but on reflection attendance may have been higher at another time fo year. These factors all may have made an impact and will be considered for furture events. As with the second event considerable use of break out rooms was made, to enable more interaction and facilitate opportunities for networking.

TO DO - evaluation of Mentoring - PH what is needed here? Is sufficient to indicate the first pilot of 10 Mentees has been established? Or do we need feedback from the participants?

To DO - evaluation of Buddying - PH what is needed here? Is sufficient to indicate the first pilot of 10 buddies has been established? Or do we need feedback from the participants?


\section{Who else has done this?}
%%Where did you get the idea from? (If from published reports, please include references). How did you find out about it? Was it easy/hard to adopt? What did you change?
As discussed in \ref{sec:Why}, in addition to generic educational challenges, many disciplines including computing have a range of discipline specific challenges. Mathematics is one such discipline and Institute of Mathematics and its Applications (IMA) has run a course for early career colleagus to help establish them in the discipline. {IMA}.

TO DO - any of disciplines doing this?

Peer to peer conversations have been reported to be a commonly used mechanism for professional development\cite{King2004}. A number of communities exist national and internationally which help promote these conversations. Internationally, groups such as ACM Special Interest Group in Computer Science Education (ACM SIGCSE) \cite{SIGCSE} or the IEEE Education Society \cite{IEEEES} promote this dialogue via conferences and other activies.  In the UK and Ireland a chapter of SIGCSE \cite{UKI-SIGCSE} further promotes these discussions by running a computer education practice conference (CEP) conference \cite{CEP} and a computing education research conference \cite{UKICER}. Much of these conversations support the professional development of colleagues who specialise in computing education. Whilst some early career colleagues will actively engage with this speciality, all will be require to develop educational capacity and effectiveness.
 
Training programmes are run by individual higher education providers. Additionally, in the UK, the Higher Education Academy (HEA) runs training programmes for academics at different stages of their career \cite{HEATraining}. However this training is not discipline specific. The Council of Professors and Heads of Computing (CPHC) run occasional workshops in a variety of issues, for example the Chair in 10 Years workshop which is aimed at facilitating career planning. Whilst these are well received contribution developmental needs are broader than those supported by these workshops.

TO DO - any other workshops?

\section{What will you do next?}
%%Will you vary this, or develop it further?

Scaling up


Funding 


Shared CPD learning materials


Address EDI issues in computer science


Reverse mentoring


Align with BCS


Special interest group

Web space for repository and functionality (mentoring, buddying) to store CPD sessions, shared resources for new CS academics, examples of good practice and a discussion space.

Explore what BCS are doing with professionals in non academic roles

Encourage research on pedagogy and educational issues in CS

Examine if programme framework is transferrable to other disciplines
\section{Why are you telling us this?}
In the UK, the community of practice related to computing education has rapidly evolved in recent years. The initiative discussed in this paper supplements the existing opportunities by:
\begin{enumerate}
	\item Networking and developmental events aimed specifically at introducing early career colleagues to one another and the wider community of computing education practice.
	\item A national mentoring programmes, running across universities providing support for early career colleagues development in research and education
	\item Providing scaffolding for naissant professional networks across universities by the establishment of a nation wide buddying scheme 
\end{enumerate}

The initiative presents significant opportunities for those of a variety of differing career stages. For those at an early career stage, the initiative presents a further source of beneficial professional development. For those with more experience, an opportunity to better understand challenges early career colleagues face as well as to grow own professional networks and to help foster the emerging community of computing education practice in the UK. 

\begin{acks}
	
	The authors would like to thank the members of the wider computing education community who have supported the initiative either be contributing to events, attending events, providing feedback, mentoring or being a mentee or a buddy.
	The BCS Academy of Computing has supported the development of this paper
	
	
\end{acks}


%%
%% The next two lines define the bibliography style to be used, and
%% the bibliography file.
\bibliographystyle{ACM-Reference-Format}
\bibliography{NewLecturer}



\end{document}
\endinput
%%
%% End of file `sample-sigconf.tex'.
