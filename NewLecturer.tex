%%
%% This is file `sample-sigconf.tex',
%% generated with the docstrip utility.
%%
%% The original source files were:
%%
%% samples.dtx  (with options: `sigconf')
%% 
%% IMPORTANT NOTICE:
%% 
%% For the copyright see the source file.
%% 
%% Any modified versions of this file must be renamed
%% with new filenames distinct from sample-sigconf.tex.
%% 
%% For distribution of the original source see the terms
%% for copying and modification in the file samples.dtx.
%% 
%% This generated file may be distributed as long as the
%% original source files, as listed above, are part of the
%% same distribution. (The sources need not necessarily be
%% in the same archive or directory.)
%%
%% The first command in your LaTeX source must be the \documentclass command.
\documentclass[sigconf]{acmart}
\usepackage[true]{anonymous-acm}
\settopmatter{authorsperrow=3}
%% NOTE that a single column version may be required for 
%% submission and peer review. This can be done by changing
%% the \doucmentclass[...]{acmart} in this template to 
%% \documentclass[manuscript,screen]{acmart}
%% 
%% To ensure 100% compatibility, please check the white list of
%% approved LaTeX packages to be used with the Master Article Template at
%% https://www.acm.org/publications/taps/whitelist-of-latex-packages 
%% before creating your document. The white list page provides 
%% information on how to submit additional LaTeX packages for 
%% review and adoption.
%% Fonts used in the template cannot be substituted; margin 
%% adjustments are not allowed.
%%
%%
%% \BibTeX command to typeset BibTeX logo in the docs
\AtBeginDocument{%
  \providecommand\BibTeX{{%
    \normalfont B\kern-0.5em{\scshape i\kern-0.25em b}\kern-0.8em\TeX}}}

%% Rights management information.  This information is sent to you
%% when you complete the rights form.  These commands have SAMPLE
%% values in them; it is your responsibility as an author to replace
%% the commands and values with those provided to you when you
%% complete the rights form.
\setcopyright{acmcopyright}
\copyrightyear{2022}
\acmYear{2022}
\acmDOI{10.1145/1122445.1122456}

%% These commands are for a PROCEEDINGS abstract or paper.
\acmConference[CEP '22]{CEP '22: ACM Computing Education Practice}{January 06, 2022}{Durham, UK}
\acmBooktitle{CEP '22: ACM Computing Education Practice Conference,
  January 06, 2022, Durham, UK}
\acmPrice{15.00}
\acmISBN{978-1-4503-XXXX-X/18/06}


%%
%% Submission ID.
%% Use this when submitting an article to a sponsored event. You'll
%% receive a unique submission ID from the organizers
%% of the event, and this ID should be used as the parameter to this command.
%%\acmSubmissionID{123-A56-BU3}

%%
%% The majority of ACM publications use numbered citations and
%% references.  The command \citestyle{authoryear} switches to the
%% "author year" style.
%%
%% If you are preparing content for an event
%% sponsored by ACM SIGGRAPH, you must use the "author year" style of
%% citations and references.
%% Uncommenting
%% the next command will enable that style.
%%\citestyle{acmauthoryear}

%%
%% end of the preamble, start of the body of the document source.
\begin{document}

%%
%% The "title" command has an optional parameter,
%% allowing the author to define a "short title" to be used in page headers.
\title{National Developmental Support for Early Career Academics in the UK Computer Science Community}

%%
%% The "author" command and its associated commands are used to define
%% the authors and their affiliations.
%% Of note is the shared affiliation of the first two authors, and the
%% "authornote" and "authornotemark" commands
\authoranon{
\author{Tom Crick}
\authornote{{\emph{N.B.}} all authors contributed equally to this paper}
\affiliation{
	\institution{Swansea University}
	\city{Swansea}
	\country{UK}
}
\email{thomas.crick@swansea.ac.uk}

\author{James H. Davenport}
\authornotemark[1]
\affiliation{%
	\institution{ University of Bath}
	\city{Bath}
	\country{UK}
}
\email{j.h.davenport@bath.ac.uk}

\author{Alan Hayes}
\authornotemark[1]
\affiliation{%
	\institution{ University of Bath}
	\city{Bath}
	\country{UK}
}
\email{ah347@bath.ac.uk}

\author{Alastair Irons}
\authornotemark[1]
\affiliation{
	\institution{ Sunderland University}
	\city{Sunderland}
	\country{UK} }
\email{alastair.irons@sunderland.ac.uk}


\author{Tom Prickett}
\authornotemark[1]
\affiliation{
	\institution{ Northumbria University}
	\city{Newcastle upon Tyne}
	\country{UK}
}
\email{tom.prickett@northumbria.ac.uk}

}

%%
%% By default, the full list of authors will be used in the page
%% headers. Often, this list is too long, and will overlap
%% other information printed in the page headers. This command allows
%% the author to define a more concise list
%% of authors' names for this purpose.
\renewcommand{\shortauthors}{Crick and Prickett, et al.}

%%
%% The abstract is a short summary of the work to be presented in the
%% article.
\begin{abstract}
Significant challenges are presented to early career academics in the United Kingdom and other jurisdictions in all disciplines and computer science academic are no exception.There are challenges in
terms of excelling at research, delivering high quality education, wider academic service commitments, as well as their
own professional development. Higher Education Institutions provide support for this by a combination of mentoring, coaching and training with early career faculty often  working towards Fellowship of the Higher Education Academy (now known as Advance HE), either by direct
application or via successful completion of an accredited
institutional taught postgraduate course). This paper reports upon progress to supplement this support from within Computing departments with an evolving UK-wider initiative to provide developmental events and to establish cross higher education institute buddying and mentoring schemes.
This paper presents a progress report regarding how this initiative has proceeded and explores the future directions under consideration, this is presented provides insights to how this initiative supplements current formal
institutional requirements with  national-level academic
practice developmental opportunities alongside national professional mentoring and buddying

\end{abstract}

%%
%% The code below is generated by the tool at http://dl.acm.org/ccs.cfm.
%% Please copy and paste the code instead of the example below.
%%
\begin{CCSXML}
<ccs2012>
   <concept>
       <concept_id>10003456.10003457.10003527</concept_id>
       <concept_desc>Social and professional topics~Computing education</concept_desc>
       <concept_significance>500</concept_significance>
       </concept>
 </ccs2012>
\end{CCSXML}

\ccsdesc[500]{Social and professional topics~Computing education}


%%
%% Keywords. The author(s) should pick words that accurately describe
%% the work being presented. Separate the keywords with commas.
\keywords{Early-career academics, community of practice, professional development, computer science education}


%%
%% This command processes the author and affiliation and title
%% information and builds the first part of the formatted document.
\maketitle

\section{What is it?}	
%%A short description of the practice you're presenting
\label{sec:What}
To do

\section{Why are you doing it?}
%%What happened before? What is it changing / replacing / improving? What gap is it filling?
To do

\section{Where does it fit?}
%%A short description of your teaching context. You may, for instance, include a description of intake, class size, curriculum sequence; anything that's necessary for others to understand your situation. How do things work at your institution?
To do
\section{Does it work?}	
%%How do you know? Give some evidence of effectiveness in context
To do

\section{Who else has done this?}
%%Where did you get the idea from? (If from published reports, please include references). How did you find out about it? Was it easy/hard to adopt? What did you change?
To do

\section{What will you do next?}
%%Will you vary this, or develop it further?
To do

\section{Why are you telling us this?}
To do
%%
%% The next two lines define the bibliography style to be used, and
%% the bibliography file.
\bibliographystyle{ACM-Reference-Format}
\bibliography{NewLecturer.bib}



\end{document}
\endinput
%%
%% End of file `sample-sigconf.tex'.
